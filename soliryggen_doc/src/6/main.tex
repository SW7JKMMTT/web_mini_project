\chapter{Deliverable VI}
\textcolor{gray}{%
\begin{itemize}
    \item Select and read one paper of your interest from the recommended reading of blocks 1 to 5. (The paper by Ian Parberry from block 3 cannot be selected.)
    \item Write a one paragraph summary of the content of the paper.
    \item Why did you choose this paper and what was the most important thing that you can perhaps use in your future professional career?
    \item What topic(s) covered during the first part of the course did you find most interesting?
\end{itemize}}

\section{Summary of \textit{How to Write an Abstract} by Philip Koopman\footnote{\url{https://users.ece.cmu.edu/~koopman/essays/abstract.html}}}
Philip Koopman argues that abstracts are vastly more important than they were a decade ago, because of the way publications and scientific work is presented though databases and search engines today.
He also presents a checklist, consisting of the parts a good abstract should have.
The checklist requires the following parts in an abstract:
\begin{enumerate*}[\itshape a\upshape)]
    \item Motivation - Why do we care about the problem and its results;
    \item Problem Statement - Defining and presenting the problem and its scope;
    \item Approach - How the problem is solved;
    \item Results - Presenting the results; and lastly
    \item Conclusions - What does this mean and how can or will it be used.
\end{enumerate*}
On top of follows said checklist one should try to limit the abstract to between 150 and 200 words, and be sure to include key points and keywords relevant to the scientific work.
Koopman concludes that writing an abstract is hard work, but a good abstract will greatly benefit the publication.

\section{Motivation}
I choose \textit{How to Write an Abstract} because the abstract is an important part of being able to reach more people with any publication.
Moreover I believe that writing a good abstract is the skill I lack the most compared to the different topics of this course to far.
The most important or helpful thing from this paper, is the checklist, which is easy to follow and provides a solid base for writing \enquote{the best} abstract in any future papers or publications of mine.

\section{Most Interesting Topics}
The most interesting topics were \textit{Topic 2: Writing a Scientific Paper} and \textit{Topic 3: Presenting a Scientific Work}.
Both of these provided usable knowledge and techniques that are directly applicable to the work we are doing in e.g.\ Semester Projects.
